% Options for packages loaded elsewhere
\PassOptionsToPackage{unicode}{hyperref}
\PassOptionsToPackage{hyphens}{url}
%
\documentclass[
]{article}
\usepackage{lmodern}
\usepackage{amssymb,amsmath}
\usepackage{ifxetex,ifluatex}
\ifnum 0\ifxetex 1\fi\ifluatex 1\fi=0 % if pdftex
  \usepackage[T1]{fontenc}
  \usepackage[utf8]{inputenc}
  \usepackage{textcomp} % provide euro and other symbols
\else % if luatex or xetex
  \usepackage{unicode-math}
  \defaultfontfeatures{Scale=MatchLowercase}
  \defaultfontfeatures[\rmfamily]{Ligatures=TeX,Scale=1}
\fi
% Use upquote if available, for straight quotes in verbatim environments
\IfFileExists{upquote.sty}{\usepackage{upquote}}{}
\IfFileExists{microtype.sty}{% use microtype if available
  \usepackage[]{microtype}
  \UseMicrotypeSet[protrusion]{basicmath} % disable protrusion for tt fonts
}{}
\makeatletter
\@ifundefined{KOMAClassName}{% if non-KOMA class
  \IfFileExists{parskip.sty}{%
    \usepackage{parskip}
  }{% else
    \setlength{\parindent}{0pt}
    \setlength{\parskip}{6pt plus 2pt minus 1pt}}
}{% if KOMA class
  \KOMAoptions{parskip=half}}
\makeatother
\usepackage{xcolor}
\IfFileExists{xurl.sty}{\usepackage{xurl}}{} % add URL line breaks if available
\IfFileExists{bookmark.sty}{\usepackage{bookmark}}{\usepackage{hyperref}}
\hypersetup{
  pdftitle={Association Rule Mining for a UK Online Retail Company},
  hidelinks,
  pdfcreator={LaTeX via pandoc}}
\urlstyle{same} % disable monospaced font for URLs
\usepackage[margin=1in]{geometry}
\usepackage{graphicx,grffile}
\makeatletter
\def\maxwidth{\ifdim\Gin@nat@width>\linewidth\linewidth\else\Gin@nat@width\fi}
\def\maxheight{\ifdim\Gin@nat@height>\textheight\textheight\else\Gin@nat@height\fi}
\makeatother
% Scale images if necessary, so that they will not overflow the page
% margins by default, and it is still possible to overwrite the defaults
% using explicit options in \includegraphics[width, height, ...]{}
\setkeys{Gin}{width=\maxwidth,height=\maxheight,keepaspectratio}
% Set default figure placement to htbp
\makeatletter
\def\fps@figure{htbp}
\makeatother
\setlength{\emergencystretch}{3em} % prevent overfull lines
\providecommand{\tightlist}{%
  \setlength{\itemsep}{0pt}\setlength{\parskip}{0pt}}
\setcounter{secnumdepth}{-\maxdimen} % remove section numbering

\title{Association Rule Mining for a UK Online Retail Company}
\author{true \and true \and true \and true}
\date{Feb 26, 2020}

\begin{document}
\maketitle

\hypertarget{abstract}{%
\subsection{Abstract}\label{abstract}}

The retail industry is one that has changed far beyond recognition with
the advent of internet. From once having to make a list, physically
travel to a store, buy \& haul your purchase yourself, to now simply
ordering your needs online and getting it delivered in less than 24
hours, retail buying has become far more convenient. This buying
convenience has also introduced challeges on the part of the sellers.
The once obvious buying patterns are no longer obvious and requires
complex analyses to understand customer preferences. In this project, we
attempt to help a UK based Online Retail store understand their
customers' buying patterns.

\hypertarget{background}{%
\subsection{Background}\label{background}}

One of the most powerful tools in online retail is a recommender system.
Such a system helps sellers mine through their sales and unearth
important associations between their products. In turn, such
associations can be presented to customers as recommendations. Our
client, the online retail store wishes to build a long term strategy
based on the understanding this project gives them.

\hypertarget{objective}{%
\subsection{Objective}\label{objective}}

The objective of our analysis is to develop an unsupervised model using
Machine Learning techniques and the CRISP-DM framework on the available
online retail data to identify \& listdown key product purchase
association patterns. This result in turn will help develop better
product recommendation.

\hypertarget{data-analysis}{%
\subsection{Data Analysis}\label{data-analysis}}

The original data set, ``Online Retail.csv'' is sourced from the UCI
Machine Learning Repository. It is a transnational data set which
contains all the transactions occurring between
01\textbackslash12\textbackslash2010 and
09\textbackslash12\textbackslash2011. The company mainly sells unique
all-occasion gifts. Many customers of the company are wholesalers.

\hypertarget{initial-data-exploration-cleaning}{%
\subsection{Initial Data Exploration \&
Cleaning}\label{initial-data-exploration-cleaning}}

The original data set in this case is a rather simple data, with
features that are obvious \& straight forward. A quick look at the
header of the data set, helps one understand this.

\scalebox{0.5}{
\begin{tabular}{lllrlrrl}
  \hline
InvoiceNo & StockCode & Description & Quantity & InvoiceDate & UnitPrice & CustomerID & Country \\ 
  \hline
536365 & 85123A & WHITE HANGING HEART T-LIGHT HOLDER &   6 & 2010-12-01 8:26 & 2.55 & 17850 & United Kingdom \\ 
  536365 & 71053 & WHITE METAL LANTERN &   6 & 2010-12-01 8:26 & 3.39 & 17850 & United Kingdom \\ 
  536365 & 84406B & CREAM CUPID HEARTS COAT HANGER &   8 & 2010-12-01 8:26 & 2.75 & 17850 & United Kingdom \\ 
  536365 & 84029G & KNITTED UNION FLAG HOT WATER BOTTLE &   6 & 2010-12-01 8:26 & 3.39 & 17850 & United Kingdom \\ 
  536365 & 84029E & RED WOOLLY HOTTIE WHITE HEART. &   6 & 2010-12-01 8:26 & 3.39 & 17850 & United Kingdom \\ 
  536365 & 22752 & SET 7 BABUSHKA NESTING BOXES &   2 & 2010-12-01 8:26 & 7.65 & 17850 & United Kingdom \\ 
   \hline
\end{tabular}
}

\newpage

We learn from the original data that all invoices that start with a
``C'' are cancelled orders and therefore are removed from the data.
Also, the patterns will be interpreted based on the product descriptions
and hence all rows with empty descriptions are deleted too. That leaves
us with data that has all valid invoices and valid descriptions

\hypertarget{models}{%
\subsection{Models}\label{models}}

From the onset, it was clear that an association rule mining model will
answer the question. In this light, it was decided to use the following
models for association rule mining

\begin{itemize}
\tightlist
\item
  Apriori algorithm
\item
  FP Growth algorithm
\item
  ECLAT algorithm
\end{itemize}

All association rule mining algorithms have 3 very important parameters
* Support * Confidence * Lift

Support: Support is the proportion that an item represents in the total
transaction dataset. Example, *support(PINK REGENCY TEACUP AND SAUCER) =
(Trasanctions featuring (PINK REGENCY TEACUP AND SAUCER))/Total
Transactions

Confidence: confidence is defined as the probability that an item
combination was bought. Example, *confidence(PINK REGENCY TEACUP AND
SAUCER \& GREEN REGENCY TEACUP AND SAUCER) = (Total transactions with
both PINK \& GREEN TEACUP \& SAUCER)/Total Transactions

Lift: Lift is defined as the increase in the chance that an item
combination is bought when a single item in that combination is bought.
Example, *Lift(PINK \& GREEN REGENCY TEACUP \& SAUCER) = Confidence(PINK
\& GREEN REGENCY TEACUP \& SAUCER)/support(GREEN REGENCY TEACUP \&
SAUCER)

\hypertarget{apriori-algorithm}{%
\subsection{Apriori algorithm}\label{apriori-algorithm}}

The apriori algorithm is one that is custom built for mining for
association rules in a dataset. This algorithm works on datasets that
are transactions. In our case, we have already converted the dataset
into ``transactions'' prior to running the model on. The apriori
algorithm, as the name suggests works on the basis of previously mined
information. It is a breadth first algorithm, where it mines for
frequent subsets by traversing across the breadth of each level of
transaction.It starts by creating a ``candidate set'' which is a table
of count of each unique item in the dataset. In the next step, it
expands by counting each frequently appearing pair and then three items
in the subsequent step and so on and so forth. At each step and finally,
the stop criterion for the algorithm is decided by the ``support''
metric explained above

\begin{verbatim}
#apriori algorithm
apriori.rules <- apriori(trans, parameter = list(supp=0.01, conf=0.8,minlen = 3, maxlen=5))
inspect(apriori.rules)
\end{verbatim}

\end{document}
